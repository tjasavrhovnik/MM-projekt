\documentclass[a4paper]{article}

\usepackage[slovene]{babel}
\usepackage{amsfonts,amssymb,amsmath}
\usepackage[utf8]{inputenc}
\usepackage[T1]{fontenc}
\usepackage{lmodern}

\usepackage{graphicx}

\begin{document}

\title{\Huge\textbf{Projekt pri predmetu Matematično modeliranje}} 
\author{\large\textsc{Tjaša Vrhovnik}\\
	Fakulteta za matematiko in fiziko\\
	Oddelek za matematiko}

\thispagestyle{empty}

\maketitle

\newpage

%%%%%%%%%%%%%%%%%%%%%%%%%%%%%%%%%%%%%%%%%%%%%%%%%%%%%%%%%%%%%%%%%%%%%%%%%%%%

\section{Naloga}

Rešujemo naslednji problem: V ravnini sta dani dve točki, $T_{1}(x_1,y_1)$ in $T_{2}(x_2,y_2)$, kjer je $y_1 > y_2$ in $x_1 < x_2$. Med vsemi kubičnimi polinomi, ki potekajo skozi točke $T_1$, $T_2$ in $T_3 = T_1 + \frac{1}{2} (T_2-T_1)$, iščemo tistega, ki minimizira čas potovanja kroglice po njegovem grafu od $T_1$ do $T_2$.

\section{Rešitev}

Naloga se zdi podobna znamenitemu problemu o brahistohroni, ki ga je zastavil Jacob Bernoulli.
Označimo iskan polinom s $p(x) = ax^3+bx^2+cx+d$. Polinom $p$ bo določen, ko bomo poznali njegove koeficiente $a$, $b$, $c$ in $d$.
Za lažje računanje za začetek prestavimo točke tako, da bo $T_1$ v koordinatnem izhodišču. Označimo nove točke:
\begin{align*}
T_{1}' &= (0,0), \\
T_{2}' &= T_2 - T_1 = (x_2-x_1, y_2-y_1), \\
T_{3}' &= T_3 - T_1 = \frac{1}{2}(T_2 - T_1) = \frac{1}{2}T_{2}'.
\end{align*}
S tem bo graf polinoma med $T_{1}'$ in $T_{2}'$ ves čas pod abscisno osjo. Če bi graf dosegel nenegativno vrednost med točkama $T_{1}'$ in $T_{3}'$, bi bil vodilni koeficient polinoma $p$ pozitiven. To bi pomenilo, da bi se kroglica iz začetne točke dvigala, kar fizikalno ni mogoče. Predvidevamo tudi, da so vrednosti polinoma med točkama $T_{3}'$ in $T_{2}'$ povsod negativne; to bomo kasneje potrdili z numeričnimi primeri.

Izračunajmo koeficiente polinoma $p$. Vemo, da je kubični polinom enolično določen s štirimi točkami v ravnini. V našem primeru so znane tri točke na polinomu, kar pomeni, da bomo imeli en prost parameter. Vstavimo koordinate točk $T_{1}'$, $T_{2}'$ in $T_{3}'$ v izraz za polinom $p$ in računajmo:
\begin{equation}
p(0) = d = 0,
\end{equation}
%
\begin{equation}
\label{Eq:p(T_{2}')}
p(x_2-x_1) = a(x_2-x_1)^3+b(x_2-x_1)^2+c(x_2-x_1) = y_2-y_1,
\end{equation}
%
\begin{equation}
\label{Eq:p(T_{3}')}
p \left(\frac{x_2-x_1}{2} \right) = a \left(\frac{x_2-x_1}{2} \right)^3+b \left(\frac{x_2-x_1}{2} \right)^2+c \frac{x_2-x_1}{2} = \frac{y_2-y_1}{2}.
\end{equation}
%
Pomnožimo enačbo~\eqref{Eq:p(T_{3}')} z $2$ in jo odštejmo od enačbe~\eqref{Eq:p(T_{2}')}. Dobimo
\begin{align}
\frac{3}{4} a(x_2-x_1)^3 + \frac{1}{2} b(x_2-x_1)^2 &= 0, \nonumber \\
\frac{3}{2} a(x_2-x_1) + b &= 0, \nonumber \\
b &= -\frac{3}{2} a(x_2-x_1).
\end{align}
%
Koeficient $c$ izračunamo iz enačbe~\eqref{Eq:p(T_{2}')}, pri čemer upoštevamo zgornji izraz za $b$. Računamo
\begin{align}
c(x_2-x_1) &= y_2 - y_1 - a(x_2-x_1)^3 - b(x_2-x_1)^2 \nonumber \\
 &= y_2 - y_1 - a(x_2-x_1)^3 + \frac{3}{2} a(x_2-x_1)^3 \nonumber \\
 &= y_2 - y_1 + \frac{1}{2} a(x_2-x_1)^3, \nonumber \\
c &= \frac{y_2-y_1}{x_2-x_1} + \frac{1}{2} a(x_2-x_1)^2.
\end{align}
%
Koeficienti polinoma $p$ se izražajo s parametrom $a$, katerega ne poznamo. Polinom $p$ je torej funkcija dveh spremenljivk, $p = p(x,a)$. Spremenljvko $a$ bomo določili z minimizacijo časa potovanja po grafu polinoma.

Čas, ki ga kroglica potrebuje za pot med točkama $T_{1}'$ in $T_{2}'$ se izraža z integralom 
\begin{equation}
\label{Eq:Cas}
t = \int_{T_{1}'}^{T_{2}'} \frac{ds}{v}.
\end{equation}
%
Upoštevajmo zakon o ohranitvi energije. V našem primeru pravi, da se vsota kinetične in potencialne energije kroglice med potovanjem ohranja. Uporabimo razmislek, da je graf polinoma ves čas pod abscisno osjo -- enačba se tako glasi
\begin{equation*}
\frac{1}{2}mv^2 = mg(-p),
\end{equation*}
od koder izrazimo hitrost potovanja kroglice
\begin{equation*}
v = \sqrt{2g(-p)}.
\end{equation*}
%
Vemo še, da je ločna dolžina enaka $ds^2 = dx^2 + dp^2$. Označimo parcialni odvod polinoma $p$ po spremenljivki $x$ z $\frac{dp}{dx}=p'$. Enačba~\eqref{Eq:Cas} tako dobi obliko
\begin{equation}
T(a) = \int_{0}^{x_2-x_1} \sqrt{ \frac{1+p'^2}{-2gp}} dx,
\end{equation}
kjer smo s $T$ označili funkcijo časa, odvisno od spremenljivke $a$. Naša naloga je poiskati njen minimum.
Analitično iskanje minimuma funkcije $T$ je zahtevno; pomagali si bomo z numeriko in s pomočjo programa Matlab poiskali vrednost $a$, ki ustreza minimalni vrednosti funkcije $T$.

Našli smo polinom, ki minimizira čas potovanja kroglice skozi točke $T_{1}'$, $T_{2}'$ in $T_{3}'$. Vendar je naša naloga poiskati polinom, ki vsebuje točke $T_1$, $T_2$ in $T_3$. Časa potovanj po grafih obeh polinomov sta enaka, saj se obliki (ukrivljenosti) polinomov ujemata. Iskani polinom namreč dobimo tako, da polinom skozi premaknjene točke transliramo za krajevni vektor točke $T_1$.
Označimo polinom skozi premaknjene točke s $p_1(x) = ax^3+bx^2+cx+d$ (njegove koeficiente smo že izračunali). Polinom skozi prvotne točke, ki nas zanima, ustreza zvezi
\begin{align}
p(x) = a(x-x_1)^3 + b(x-x_1)^2 + c(x-x_1) + d + y_1.
\end{align}
Ker poznamo koeficiente $a$, $b$, $c$ in $d$, vrednosti $x_1$ in $y_1$ pa sta koordinati točke $T_1$, poznamo tudi iskani polinom $p$.

%%%%%%%%%%%%%%%%%%%%%%%%%%%%%%%%%%%%%%%%%%%%%%%%%%%%%%%%%%%%%%%%%%%%%%%%%%%%
% Literatura

\begin{thebibliography}{99}

\bibitem{zapiski} Zapiski s predavanj prof.~E.~Žagarja pri predmetu Matematično modeliranje, Univerza v Ljubljani, Fakulteta za matematiko in fiziko (študijsko leto 2018/2019).

\end{thebibliography}

\end{document}